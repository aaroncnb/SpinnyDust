\chapter*{Abstract}
\addcontentsline{toc}{chapter}{Abstract}
The anomalous microwave emission (AME) still lacks a coherent explanation.  This excess of emission, roughly betweewn 10 and 50 GHz, tends to defy attempts to explain it as synchrotron or free-free emission. The overlap with frequencies important for cosmic microwae background explorations, combined with a strong correlation with interstellar dust, drive cross-disciplinary collaboration between interstellar medium and obervational cosmology. The apparent relationship with dust has prompted a ``spinning dust''hypothesis:  electric dipole emission by rapidly rotating, small dust grains. Magnetic dipole emission by grains with magnetic inclusions (``magnetic dust'') while less suppported, has not been ruled out. Even assuming a spinning dust scenario, we are far from concluding which population of dust would be contribution. However the typical peak frequency range of the AME profile implicate grains on the order of ~1nm. This points to polycyclic aromatic hydrocarbon molecules (PAHs)/ We use data from the AKARI/Infrared Camera (IRC), due to its thorough PAH-band coverage, to compare AME (Planck Coll. astrophysical component separation product) with infrared dust emission. The results and discussion contained here apply to an angular scale of approximately 1$^{\circ}$. In general, our results support an AME-from-dust hypothesis. At 1$^{\circ}$ angular resolution, we do not find evidence that the AME is exclusively carried by PAHs. The correlation of far IR dust emission with AME appears to be the best predictor, with MIR/PAH emission being marginally more weakly correlated with AME. In the case of lambda Orionis, AME vs. PAH is at least as strong as AME vs. FIR. Higher resolution studies will be needed in the future to fully solve the AME mystery. re consistent with previous studies in that the AME has a clear connection to interstellar dust, but a conclusive link to any particular population of dust is unapparent.
