\chapter*{Abstract}
\addcontentsline{toc}{chapter}{Abstract}
The anomalous microwave emission (AME) still lacks a conclusive explanation.  This excess of emission, roughly between 10 and 50~GHz, tends to defy attempts to explain it as synchrotron or free-free emission. The overlap with frequencies important for cosmic microwae background explorations, combined with a strong correlation with interstellar dust, drive cross-disciplinary collaboration between interstellar medium and obervational cosmology. The apparent relationship with dust has prompted a ``spinning dust'' hypothesis:  electric dipole emission by rapidly rotating, small dust grains. Magnetic dipole emission by grains with magnetic inclusions (``magnetic dust''), while less suppported, has not been ruled out. Even assuming a spinning dust scenario, we are far from concluding which category of dust contributes. The typical peak frequency range of the AME profile implicates grains on the order of ~1nm. This points to polycyclic aromatic hydrocarbon molecules (PAHs). We use data from the AKARI/Infrared Camera (IRC; \citet{irc07}), due to its thorough PAH-band coverage, to compare AME from the \cite{planck15X}. astrophysical component separation product) with infrared dust emission. We look also at infrared dust emission from other mid IR and far-IR bands. The results and discussion contained here apply to an angular scale of approximately 1$^{\circ}$. In general, our results support an AME-from-dust hypothesis. We look both at $\lambda$~Orionis, a region highlighted for strong AME, and find that certainly dust mass correlates with AME, and that PAH-related emission in the AKARI/IRC 9~$\mu$m band may correlate slightly more strongly. These results are compared to an all-sky analysis, where we find that potential microwave emission component separation imperfections among other issues, make an all-sky, delocalized comparsion very challenging. In any case the AME-to-dust correlation persists even in the all-sky case, but tests of relative variations from different dust SED components are largely inconclusive. We emphasize that future efforts to understand AME should focus on individual regions, and a detailed comparsion of the PAH features with the variation of the AME SED. Further all-sky analyses seem unlikely to help resolve this issue. Non-PAH carriers of the AME, such as nanosilicates, cannot be ruled out either.
