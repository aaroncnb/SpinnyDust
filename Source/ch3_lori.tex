\chapter{Analysis of an interesting AME region: $\lambda$~Orionis}
\label{ch:lori}

 \section{An interesting AME region}
		The $\lambda$~Orionis molecular ring, also known as the Meissa Ring is a massive stucture surrounding the $\lambda$~Orionis O-type star. The ring contains an HII region, ionized by LOri itself and its OB associates \citep{murdin77}. What had been thought of as a star\-forming region of missing molecular gas. At the time \cite{murdin77} even speculated that this could be evidence of an alternate star\-formation pathway, writing: ``Notably we need to know if $\lambda$Ori is an example of a different mode of star formation or [...] simply a case in which the progenitor molecular cloud was exhausted within the last one or two million years.''

    \cite{maddalena86,maddalena87}.  (and references therein) noted a ring of material likely being pushed out by the central, historically well-known $\lambda$~Orionis Association of B-type stars and surrounding HII reigon.

   \subsection{Where does the ring come from?}
    \cite{cunha96} argued that the ring may have resulted from a supernova explosion, further speculating that LOri may have been a companion of the progenitor. $\lambda$~Orionis is a known binary system, however its current companion is the a B-type star. \citep{murdin77}
  .
     The central region is heated by the $\lambda$~Orionis star itself, and the Orion~OB association it belongs to \citep{ochsendorf15}. The region is known to host several young stellar and protostellar objects \citep{koenig15}.

    At approx. 10$^{\circ}$ wide, we can see the outline of the structure even in the low (1$^{\circ}$ FWHM) resolution PCAME map. The ring shape itself is thought to originate from a supernova, or perhaps combined effects of the entire star formation history of the $\lambda$~Orionis Association, including the formation of its surrounding HII region \citep{aran09}.

  \subsection{A well-studied region}
    Although the $\lambda$~Orionis region has been a popular target for study since approximately the 1980s.\cite{duerr82} wrote of the relative lack of work on the overall region: ``Surprisingly, this interesting complex has been little studied''. While this seems surprising given the numbe of works on the region in the literature now, it is really the advent of all-sky missions that have driven more recent interest.  The large angular size is such that all-sky surveys were a natural boon for study of such extended structures. WISE especially was a huge source of insight \citep{koenig15}. More recently, \cite{planck15XXV} strongly highlighted the region as a strong candidate for further AME investigation.

\section{Investigative approach}
  We have carried out an initial comparison of the AME of this region with its mid to far-IR dust emission. The region is shown in Fig.~\ref{fig:orionis-akari9} as it appears in 1$^{\circ}$-smoothed A9 data.
      \begin{figure}
        \includegraphics[width=\textwidth]{../Plots/LOri_akari9_AMEcont_1dres.pdf}
        \centering
        \caption{$\lambda$~Orionis as it appears in the AKARI 9~$\mu$m data. Contours indicate the AME, as given by the Planck PR2 AME map. The image is smoothed to a 1$^{\circ}$ PSF (much larger than the original 10 arcsec map). The $\lambda$~Orionis star itself is approximately located at the center of the image.}
        \label{fig:orionis-akari9}
      \end{figure}
   The ring structure itself indicates excess microwave emission attributed to AME, from the dominant variable frequency component $AME_{var}$ (see Ch.~\ref{ch:datasources}). The central region is dominated by free-free emission \citep{aran09, koenig15}. Free-free emisison coming from the Hii phase surrounding the $\lambda$~Orionis association dominates the region's morphology in LFI images. \citep{planck15XXV}. Taking the hint from \cite{planck15XXV} that this may be among the more reliably component separated regions, we evaluate if there is any preferential relationship between any parameter of dust emission and the AME. Fig.~\ref{fig:LOri_halpha_AMEvarContours} shows the expected distribution of free-free emission in the region, assuming that $H\alpha$ line emission is a tracer of microwave free-free. This indicates that free-free is strong within the central region, where radiation fields are more intense, and AME is minimal. The strongest AME follows the ring-shaped morphology outside the central bubble of free-free emission.
     \begin{figure}
       \includegraphics[width=\textwidth]{../Plots/ch_lori/LOri_halpha_AMEvarContours.pdf}
       \centering
       \caption{$\lambda$~Orionis as it appears in H-alpha emission by \cite{finkbeiner03}. Contours indicate AME emission, from the variable frequency component. The colorbar indicates $H\alpha$ emission in Rayleighs. The field of view is slightly larger than that used for the final processed IR comparison (Fig.~\ref{fig:lori_processed_all}). }
       \label{fig:LOri_halpha_AMEvarContours}
     \end{figure}

	\section{Data preparation}
    \label{sec:dataprocessing}
		As indicated in Ch.~\ref{ch:datasources}, we use 12 photometric all-sky maps. For the IRC data (A9 and A18), we produce mosiacs of $\lambda$~Orionis from the individual tiles provided in the internal all-sky archive.
      \footnote{IRC all-sky data is still in the proprietary phase at the time of this writing, but should be public by April 2018.}
       For the other sources, HEALPix all-sky maps are available publicly, at sufficient resolution relative to their native resolutions. \footnote{Planck data was retreived from the NASA IPAC online archive at \url{http://irsa.ipac.caltech.edu/data/Planck/release_2/all-sky-maps/}}\footnote{AKARI/FIS data }\footnote{IRAS/IRIS data }

		\subsection{Extraction from HEALPix maps}
		  For the data obtained via HEALPix maps, we employ the ``healpix2wcs'' functionality provided in the ``gnomdrizz'' python package\footnote{Available at \url{http://cade.irap.omp.eu/dokuwiki/doku.php?id=software}}\footnote{``drizzlib'' 1.2.2 and earlier were not able to correctly access HEALPix files with multiple fields/columns. See appendix for our recommended workaround.} A9 and A18 images are produced by regridding the images with the ``montage'' software by NASA/IPAC. Figs.~\ref{fig:lori_A9_mosaic_smooth} and~\ref{fig:lori_A18_mosaic_smooth} show high resolution mosaics of the A9 and A18 data before processing). All of the images for all of the bands are based on a common FITS header which has a pixel grid spacing equal to the average pixel width in the NSIDE 256 HEALPix scheme.

      \subsection{Point-source and artifact masking}

        \paragraph{Missing stripes}
          The AKARI all-sky survey suffers from a few missing stripe errors throughout the IRC and FIS maps \citep{ishihara10, doi15}. These are more a more serious issue for the far infrared maps. They can generally be characterized as a competition for integration time beteen the all-sky observing mode, and pointed observations during AKARI's crygoenic phase. Unfortunately for the present work, some of these stripes pass directly through the $\lambda$~Orionis region. Figs.~\ref{fig:LOri_FIS_color},~\ref{fig:lori_A9_mosaic_smooth} and ~\ref{fig:lori_A18_mosaic_smooth} display the data at near-native resolution, demonstrating where these patterns occur. Additionally there are some saturated pixels in both IRC and FIS data.
            \begin{figure}
              \includegraphics[width=\textwidth]{../Plots/ch_lori/lori_fis_rgb.pdf}
              \centering
              \caption{The $\lambda$~Orionis region and its surroundings in AKARI FIS data, where A65 is blue, A90 is green, and A140 is red. The missing stripe patterns are visible, and affect all 3 bands shown here as well as the A160 data. }
              \label{fig:LOri_FIS_color}
            \end{figure}
            \begin{figure}
              \includegraphics[width=\textwidth]{../Plots/ch_lori/lori_A9_mosaic_smooth.pdf}
              \centering
              \caption{The $\lambda$~Orionis region in the A9 band at near-native resolution. This is a mosiac created from the 3x3 degree all-sky survey tiles by Ishirara et al. (in prep.) Missing stripes are less of a problem than with the A18 band (Fig.~\ref{fig:lori_A18_mosaic_smooth}), but point sources are more pronounced. Betelgeuse, in the lower left of the image, is bright enough in this band to produce a ring-shaped artifact. }
              \label{fig:lori_A9_mosaic_smooth}
            \end{figure}
            \begin{figure}
              \includegraphics[width=\textwidth]{../Plots/ch_lori/lori_A18_mosaic_smooth.pdf}
              \centering
              \caption{The $\lambda$~Orionis region in the A18 band at near-native resolution, demonstrating the regions affected by missing stripe errors, but less affected by point-sources than A9 (Fig.~\ref{fig:lori_A9_mosaic_smooth}). }
              \label{fig:lori_A18_mosaic_smooth}
            \end{figure}

        \paragraph{Point sources}
          A caveat that comes with added ionized PAH feature coverage of the A9 band, is that the shorter central wavelength placement allows more contamination from point sources. We identify point sources with a moving-window approach provided in the astropy Python package, flagging pixels which have higher than $5~\sigma$ intensity among the surrounding 100 pixel window. We then place a mask at the center of the flagged point-sources. The masks are propogated through the regridding step, such that the low-resolution pixels having more than 50\% of their area masked in the high resolution tiles, also become masked. Such pixels appear black in Fig.~\ref{fig:orionis-akari9}. The same process is applied to the A18 data. For other maps, which we extract from HEALPix data, we first regrid from HEALPix to rectangular grids, and then apply the point-source search and masking as above. For the I12 and I25 images, the rejected pixels were fewer than with A9, but positions overlapped with those already masked in A9. For bands longer than I25, this process did not result in rejected pixels. For the D12 and D25 bands, which are natively at a much lower resolution than AKARI or IRAS, point-sources present more of a challenge. For these images, we visually inspect and mask 3 regions with bright point source contamination consistent with the DIRBE beamsize and with point-sources identified in IRAS and IRC images.


        \subsection{PSF Smoothing}
          We smooth the pixels in the spatial domain, to have a 1 degree FWHM PSF, in order to have a resolution approximating that of the PC AME data.
          The smoothing process relies on the {\tt convolution} module provided in the ``astropy'' package. We assume simple circular gaussian kernel for the smoothing process. While they may be asymmetries in the effective beam shapes of the IR bands used, the target resolution of the AME data is large enough relative to the native resolution of the input IR data (especially A9 and A18, see Fig.~\ref{tab:data}) as to render the beam shapes and positional variations negligible. Finally, we mask pixels along the edge of the FOV where the convolution process produces artifacts.

      \subsection{Background subtraction}
         We estimate an average, flat background level for this region. The background level is determined the mean of pixels in an `OFF` zone. The final images are shown in Fig.~\ref{fig:lori_processed_all}, with the full mask applied (masked pixels are indicated in white), and with the OFF zone indicated by the red rectangle on each frame.
          \begin{figure}
            \includegraphics[width=\textwidth]{../Plots/ch_lori/lori_processed_grid.pdf}
            \centering
            \caption{Processed data at each wavelgnth for $\lambda$~Orionis. A flat background has been subtracted from each frame based on the mean of pixels within the red rectangle. The pixel width is 0.25 degrees, with the data PSF smoothed to 1 degree spatial resolution. Masked pixels (point sources, stripe errors, convolution artifacts) are shown as white. The frames share the same FOV and GAL-TAN projection. Colorbars indicate the intensity in MJy/sr.}
            \label{fig:lori_processed_all}
          \end{figure}
          We do not expect simple band-by-band intensity correlation tests with the AME to be sensitive to background and foreground emission along the line of sight towards the $\lambda$~Orionis region. However analyses such as dust SED fitting, to determine the relative abundances of different dust components, may be effected by the background level. The general morphology as seen in the high resolution AKARI data (Figs.~\ref{fig:LOri_FIS_color},~\ref{fig:lori_A9_mosaic_smooth} and ~\ref{fig:lori_A18_mosaic_smooth}) remains well pronounced in the final, low resolution images.

	\section{Multi-wavelength characterization}
    The correlation matrix results corresponding to data shown in Fig.~\ref{fig:lori_processed_all}, are shown in Fig.~\ref{fig:orionis-corr-matrix}.
      \begin{figure}
        \includegraphics[width=\textwidth]{../Plots/ch_lori/Lori_corrmatrix_I.pdf}
        \centering
        \caption{$r_{s}$ correlation matrix for all of the data used in the $\lambda$~Orionis analysis, similar to that presented for the Planck Commander component maps in Fig.~\ref{fig:PCCS_corrmatrix}. The shade and annotation for each cell indicates the $r_{s}$ score, where $r_{s}$ of 1 indicates a monotonically increasing relationship for a given pair of images. The two AME components, as described in Ch.~\ref{ch:datasources}, are listed separately: $AME_{var}$ for the frequency-varying component, and $AME_{fix}$ for the constant frequency component.}
        \label{fig:orionis-corr-matrix}
      \end{figure}
     This immediately confirms a correlation between the IR and AME, however this was readily visibly from the spatial morphology of the region. Interestingly though, the correlation strengths with $AME_{var}$ show a pattern from short to long wavelengths: A9, P857, and P545 show the strongest correlations, with the correlation weaking from A18 to A90, and again strengthening at longer wavelengths. The fixed peak frequency $AME_{fix}$, which is the much fainter component, shows the strongest correlation with A90- though all of the IR correlations relative to $AME_{fix}$ are weaker than those for $AME_{var}$. The overall pattern is for bands dominated by PAH emission (as discussed in Ch.~\ref{ch:datasources}), and those which trace Rayleigh-Jeans thermal dust emission are equally good predictors of the AME. Bands dominated by a mixture of VSGs, and warm dust emission, show a weaker correlation. For the next stages of analysis, we will consider only the dominant component $AME_{sum}$, however the results are not sensitive the choice of $AME_{sum}$ or $AME_{var}$.

     Comparing the images in Fig.~\ref{fig:lori_processed_all}, most of the variation in the correlation scores appears to come from the central region of $\lambda$~Orionis. Because of the known heating present within the ring, from the $\lambda$~Orionis association, and given the brightening of bands between A18 and A90, this variation appears to be due to a temperature increase.
    \subsection{Bootstrap analysis}
        To assess the robustness of the correlation scores, we employ the Bootstrap re-sampling approach, first introduced by \cite{efron79} (see \cite{feigelson13} for an updated description within an astronomical context). This involves creating random re-sampled sets of the data. We use the 'with replacement' approach, meaning that a data point may be selected multiple times in a single re-sampling iteration. The size of the re-sampled set is the same is the input set size. For each random set we run a correlation test, resulting in a distribution of correlation coefficients. This allows us to estimate error bars for the correlation scores, and effectively de-weight outliers within each sample.
        We carry out bootstrap correlation tests for each IR band's intensity vs. the $AME_{sum}$ integrated intensity. The data are resampled 10,000 times for each correlation test, a sufficient resampling given the unmasked pixel count of \~1400 pixels, and considering that the effective beams are somewhat undersampled. This is repeated for both $r_{s}$ and $r_{p}$ based tests, in order to assess variations in degree of linearity among the various IR:AME trends. The distributions of the boostrap resamplings are shown in in Fig.~\ref{fig:bootstrap_vs_AME}.
            \begin{figure}
              \includegraphics[width=\textwidth]{../Plots/ch_lori/bootstrap_vs_AME_pearson_i10000.pdf}
              \includegraphics[width=\textwidth]{../Plots/ch_lori/bootstrap_vs_AME_spearman_i10000.pdf}
              \centering
              \caption{Re-sampled (Bootstrap) correlation tests for IR emission in $\lambda$~Orionis vs. AME, for both $r_{p}$ (upper) and $r_{s}$ (lower) cases. Each band's $r$ distribution is shown in a different color (the same color scheme for both plots). The width of the distribution indicates the error for the given data in the correlation coefficient. The mean and standard deviation of the scores are given in the legend of each plot. The plot ranges only show positive values, since no negative scores were produced. }
              \label{fig:bootstrap_vs_AME}
            \end{figure}
        For both test cases, the best correlations are the longest and shortest wavelength bands, consistent with the straight-forward $r_{s}$ scores shown in Fig.~\ref{fig:orionis-corr-matrix}. In the $r_{p}$ case, the strongest correlation is the A9 band vs the AME.
        \section{Comparison with SED Fitting}
          We performed a full dust SED fitting on the $\lambda$~Orionis photometry, according to the dust model by \cite{galliano11} (to be thoroughly introduced in Galliano, et al., in prep.)  We used a mixture of silicate and carbonaceous dust, silicate dust, the two dominant categories of interstellar dust as described in Ch.~\ref{ch:intro}. However, instead of the graphite-based carbon dust invoked by the canonical \cite{draine07} model (DL07), we assume amorphous carbon. This choice is an attempt to account for the ``sub-mm excess'' of dust emission, reported by \cite{israel10, bot10}, in the Large Megallanic Cloud (LMC). The increased emissivity of amorphous carbon (a factor of 2-3 more than DL07) allows a better fit to Herschel observations of the LMC \citep{galliano11}, and Planck observations of the Milky Way \citep{planckIntXXIX16}. We assume that the radiation field heating this dust mixture is the Galactic ISRF \citep{mathis83}, scaled by a factor $U$.
        %   We also assume, following \cite{dale01}, that the dust is exposed to a distribution of starlight intensity, distributed as:
        %       \begin{equation}
        %          \label{eq:U}
        %            dM_{dust}\propto{} U^{-\alpha}dU
        %       \end{equation}
        %   between $U_{min}$ and $U_{max}$, where $U_{min}$, $U_{max}$ and $\alpha{}$ are free parameters.
          We utilize both the Bayesian dust SED fitting approach by Galliano (in prep.), and a least-squares analysis. The SED fitting accounts for estimates of the respective calibration uncertainties, and the spectral response curves (Fig.~\ref{fig:Filter_coverage_example_full}). As outputs, metrics of the total dust mass $M_{dust}$, PAH mass $M_{PAH}$, ionized PAH mass $M_{PAH+}$, and ISRF intensity $U$ are produced. We only carry out the fitting for unmasked pixels. We are primarily interested in which of the corrleations $M_{dust}$ vs. $I_{AME}$ or $M_{PAH}$ vs. $I_{AME}$ is stronger.
          Two sample fitting results are shown in Figs.~\ref{fig:fred_LOri_notes_Oct2017_fig1a}, and~\ref{fig:fred_LOri_notes_Oct2017_fig1b}.
              \begin{figure}
                \includegraphics[width=\textwidth]{../Plots/ch_lori/fred_LOri_notes_Oct2017_fig1a.pdf}
                \centering
                \caption{Observed (black circles and errors) and synthetic photometry (gray dots) SED of a pixel within $\lambda$~Orionis, along with the dust SED model fit results. Two SED fits are shown: on for the Bayesian fitting (magenta), and another showing the standard least-squares result for comparison (yellow). The fitted ISRF strength $U$, and fraction of mass in PAHs, $q_{PAH}$ are also given.}
                \label{fig:fred_LOri_notes_Oct2017_fig1a}
              \end{figure}
              \begin{figure}
                \includegraphics[width=\textwidth]{../Plots/ch_lori/fred_LOri_notes_Oct2017_fig1b.pdf}
                \centering
                \caption{The same as Fig.~\ref{fig:fred_LOri_notes_Oct2017_fig1a}, but for a different pixel position. Corresponds to galactic coordinates...}
                \label{fig:fred_LOri_notes_Oct2017_fig1b}
              \end{figure}
           Performing such fits for all of the pixels, we are able to see how $I_{AME}$ varies with the bulk dust physical characteristics of the region. Fig.~\ref{fig:fred_LOri_notes_Oct2017_fig2a} shows the fitted dust mass per pixel, relative to the AME intensity.
                \begin{figure}
                 \includegraphics[width=\textwidth]{../Plots/ch_lori/fred_LOri_notes_Oct2017_fig2a.pdf}
                 \centering
                 \caption{Scatter plot with error elipses generated through the Bayesian SED fitting, of total dust mass $M_{dust}$ vs. $I_{AME}$ scaled by $U$.}
                 \label{fig:fred_LOri_notes_Oct2017_fig2a}
               \end{figure}
           AME intensity is scaled by the ISRF intensity $U$. Although spinning dust emission is not predicted to vary directly with $U$, we consider that the ISRF may serve as a diagnostic of environmental conditions in the ISM. In any case, we find that performing such a scaling improves the correlations with dust mass.   Figs.~\ref{fig:fred_LOri_notes_Oct2017_fig2b} and~\ref{fig:fred_LOri_notes_Oct2017_fig2d} describe the variation with $M_{PAH}$ and $M_{PAH+}$.
              \begin{figure}
                \includegraphics[width=\textwidth]{../Plots/ch_lori/fred_LOri_notes_Oct2017_fig2b.pdf}
                \centering
                \caption{The same comparison is given by~\ref{fig:fred_LOri_notes_Oct2017_fig2a}, but showing total mass of PAHs ($M_{PAH}$) rather than total dust mass on the y-axis. }
                \label{fig:fred_LOri_notes_Oct2017_fig2b}
              \end{figure}
              \begin{figure}
                \includegraphics[width=\textwidth]{../Plots/ch_lori/fred_LOri_notes_Oct2017_fig2d.pdf}
                \centering
                \caption{ The same as in Figs.~\ref{fig:fred_LOri_notes_Oct2017_fig2a} and~\ref{fig:fred_LOri_notes_Oct2017_fig2b} , but specifically comparing an estimate of the charged component of PAH mass $M_{PAH+}$. This includes anions and cations, since we cannot distinguish between these two spectroscopically.}
                \label{fig:fred_LOri_notes_Oct2017_fig2d}
              \end{figure}
    Based on the dust properties derived from these SED fits, we attempt investigate whether any fitted parameter shows a preferential relation with the AME.

    \subsection{Comparison with unmasked results}
        To assess the effect of applying the mask and background subtraction from Sec.~\ref{sec:dataprocessing}, Figs.~\ref{fig:orionis-img} and \ref{fig:orionis-corr} show a very simplistic version of the image processing. No point-source masking, background adjustment, or missing strip masking are applied. Data are simply smoothed with a 1-degree Gaussian beam, and missing data are interpolated.
          \begin{figure}
            \includegraphics[width=\textwidth]{../Plots/lOrionis_grid_img.png}
            \centering
            \caption{A grid of thumbnails showing the $\lambda$~Orionis region's structure, at 12 wavelengths, along with AME contours (shown in white countours. Spatial correlation seems to be the best at the shortest and longest wavelengths (AKARI/IRC 9~$\mu$m and Planck/HFI 550~$\mu$m). The images are only smoothed and interpolated, for demonstration of the effects without other image processing steps (i.e. background subtraction, point source masking). Figure~\ref{fig:orionis-akari9} demonstrates the actual pixel grid used for the SED fitting and intensity correlation tests.}
            \label{fig:orionis-img}
          \end{figure}
          \begin{figure}
            \includegraphics[width=\textwidth]{../Plots/orionis_correlations_AME.png}
            \centering
            \caption{Intensity cross-correlation for all pixels in the $\lambda$~Orionis cut-out region.  $r_{s}$ indicates the Spearman rank correlation coefficient for each plot.}
            \label{fig:orionis-corr}
          \end{figure}
        Contours indicate the region's shape in the PCAME map. Figure~\ref{fig:orionis-corr} shows IR to AME cross correlation plots, for all pixels within the 10$^{\circ}$ by 10$^{\circ}$ $\lambda$~Orionis region. Using this more crude processing, we see in Fig. find the same general pattern--- A9 and P545 show the tightest correlation. Thus the overal result, in terms of a ranking of correlation strengths, does not seem to depend heavily on the image processing steps applied.

  \section{Discussion}
      In  $\lambda$~Orionis we found that accross the whole region, A9 emission and P545 emission were the most strongly correlated with AME. This is apparent both in the photometric band analysis, and in the dust SED fitting.  The fact that the correlation strengths of PAH-tracing mission and sub-mm emission are similar is in-line with what we have seen in \cite{ysard10b} and \cite{hensley16}. In those works, the two relationships (MIR vs. AME and FIR vs. AME) are very close, although these two papers are odds as to which relationship is stronger, and thus in their final interpretation. With the present data and analysis of $\lambda$~Orionis, we fail to rule out PAHs as carriers of the AME. Fig.~\ref{fig:fred_LOri_notes_Oct2017_fig2c} indicates that although total dust mass and PAH mass are both correlated with AME, there is a strong (\~100\%) probability that PAH mass is the stronger predictor of AME intensity.
          \begin{figure}
            \includegraphics[width=\textwidth]{../Plots/ch_lori/fred_LOri_notes_Oct2017_fig2c.pdf}
            \centering
            \caption{ The Bayesian correlation probablity distributions of Pearson's $\rho{}$) for the three physical parameters vs. the AME intensity: total dust mass, $rho_{dust}$ (red); total PAH mass $rho_{PAH}$ (green); and only the ionized PAH mass $rho_{PAH+}$ (blue). Also given are the probabilities of either PAH component being better correlated with AME than dust mass, as well as the probability that ionized PAH mass correlates better than total PAH.}
            \label{fig:fred_LOri_notes_Oct2017_fig2c}
          \end{figure}
        The results are consistent with a scenario in which PAH mass, cold dust, and the AME are all tightly correlated. A correlation between cold dust and PAHs is observed in extragalactic targets \citep{haas02}. In the case that AME emanents from spinning PAHs, it is not surprising that cold dust would also correlate with the AME. Weaker correlation from 25 to 70~$\mu$m may indicate that AME is weaker in regions of warmer dust and stronger radiation fields. Such an anti-correlation with harsher radiation are consistent with the carriers of AME being destroyed in the central region of $\lambda$~Orionis, thus leading to substantially decreased spinning dust emission.

      \subsection{PAH Ionization fraction}
          As described in Ch.\ref{ch:datasources} it is expected that relative variations between the A9 and I12 intensities could be explained by the fraction of PAHs that are charged, $fPAH+$. Spectroscopically, we cannot distinguish between PAH anions or cations. However if spinning dust emission arises from anions, a better correlation with the mass of charged PAHs $M_{PAH+}$ is expected. However if the PAHs are positively charged, a stronger correlation with $M_{PAH+}$ is not expected. This is due to the rotational excitability of the PAHs: anions are more succeptable to rotational excitation by $H^{+}$ and $C^{+}$ collisions\citep{ali-haimoud10}.

          Examining $\lambda$~Orionis in intensity, we find that the A9 intensity correlates more strongly with AME than I12 or D12. In the $r_{p}$ case, A9 correlates more strongly with AME than any other band. This is consistent with the spinning PAH hypothesis, and taken alone may indicate that the 6.2~$\mu$m feature emission from charged PAHs, may be a better predictor of AME intensity.

          As shown by the dust SED fitting however, the probability distributions (Fig.~\ref{fig:fred_LOri_notes_Oct2017_fig2c}) of $r_{p}(M_{PAH+}:I_{AME})$ do not indicate that ionized PAH mass correlates better with the total PAH mass. Attempts to estimate the $M_{PAH+}$ based on the available data appear to only add noise relative to $r_{p}(M_{PAH}:I_{AME})$. The means of the two distrubtions $r_{p}(M_{PAH+}:I_{AME})$  and $r_{p}(M_{PAH}:I_{AME})$ are similar and $r_{p}(M_{PAH+}:I_{AME})$ shows a wider distribution. Thus the question of whether or not AME comes predominantly from charged PAHs remains open.

          The fact that A9 correlates more strongly than the 12~$\mu$m bands, at least suggests that this topic is worth further investigation. What is clear from the MIR and AME morphology however, is that there is a transition from a relatively PAH depleted, warmer, stronger ISRF in the centerand warm dust in the center to a PAH-supporting region in the ring. Along this transition, there must be a decreasing radiation field with distance from dimnant heating sources, in $\lambda$~Orionis association. \cite{andrews16} predict a transition of PAH species along such a radiation field gradient, from complete PAH destruction in harsh environments, to survival of (sufficiently large) PAH anions near the surface of molecular clouds. Thus if our stronger correlation with A9 indicates charged PAHs, this could be consistent with PAH anions surviving in the portions of $\lambda$~Orionis which are emitting the strongest AME. Future wide-area spectral mapping of the $\lambda$~Orionis region may be able to conclusively test for increased $fPAH+$ in regions with stronger AME. Such studies would eb strongly aided by higher resolution probing of spatial variations the AME spectral profile.
