\chapter{Summary}

  From these results, we cannot confirm or rule out a spinning-PAH hypothesis on an all-sky, 1$^{\circ}$ scale. While nanosilicates or magentic dipole emission from dust may be plausible contributors to the AME, as shown in \cite{hoang16a, hensley17a}, we do not find (at least at resolutions considered here) that PAHs can be ruled-out as a carrier. While it is true that the FIR in our study has a tighter correlation with AME than PAH-related bands, the correlation with PAH emission remains. This is true both when considering AME intensity to Either this is a coincidental correlation- PAHs and AME are both correlated with the ``actual'' carrier/s of AME, MIR phometric bands are not as good of a tracer of the actual PAH mass as we believed, or perhaps that PAHs are indeed contributing to the AME but that they are just one of multiple sources of the AME.

   If it can be shown that IR emission from non-PAH small dust particles, such as nanosilicates strongly correlates with the AME, another very interesting question would need to be addressed: why does PAH emission \textit{not} show a strong correlation? Do PAHs not have the range of dipole moments needed to produce emission in the 10 - 90 GHz range? Because, as is described in \cite{draine98a} and again in \cite{hensley17b}, such small particles should be spinning at the frequencies consistent with AME. Thus if we believe a particular class of small dust particles to exist, PAHs, nanosilicates, or otherwise- then they must be producing microwave emission, unless they do not have a permanent electric dipole.

  If there is a sufficient abundance of PAHs with an electric dipole, then we must consider the possibility that the data currently available to do not offer the necessary spatial resolution, or that the photometric bands used do not allow us to adequately separate the individual dust components (i.e. the PAH features from potential nanosilicate features) and/or microwave foreground emission components (free-free from spinning-dust.) We look forward to continued environment-resolved comparisons to investigate the potential AME-PAH (or AME-nanosilicates, iron nanoparticles) relationship in a region by region, especially given the disagreement we find between our examination of lambda Orionis and the all-sky analysis.


 This project is supported by JSPS and CNRS under the Japan--France Research Cooperative Program. We would like to give special thanks to the AME Workshop 2016 attendees and organizer, Chris Tibbs, for enlightening disucssions. Thanks also to Nathalie Ysard and Steven Gibson, and the staff of Grid, Inc. for helpful feedback.
