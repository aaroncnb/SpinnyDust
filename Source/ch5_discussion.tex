\chapter{Discussion}
  \label{ch:discussion}

    We have compared AME with infrared dust emission from 2 approaches. The results and discussion contained here apply to an angular scale of approximately 1$^{\circ}$. In general, our results support an AME-from-dust hypothesis. We do not find evidence that the AME is exclusively carried by PAHs, but we do find evidence that the AME correlates with the AKARI 9 micron band emission better than other MIR bands- and better than all other IR bands sampled, when looking at the particularly prominent AME region, $\lambda$~Orionis. The results may be consistent with a scenario in which PAH mass, cold dust, and the AME are all tightly correlated.

\section{Future Works}
    We note two major limitations in the current state of AME, and in general, all multi-wavelength analysis.
    The first is data limitations, the second is a lag in analysis innovations.

    In terms of data, there are two major boundaries that, based on results presented here, must be pushed. Spatial resolution and spectral coverage.
    Spatial resolution constraints are critical issue in multiwavelength astornomy, especially in dust-related works. To explain why, simply consider the description of dust researhc from Ch.~\ref{ch:intro}, and Tab.~\ref{tab:data}. The spatial resolution of available data has a wide discrepancy from the MIR to microwave: ~10 arcseconds for AKARI 9 micron to at least a degree FWHM for the ``effective resolution'' of parameter maps. This is a profound discrpenancy for all-sky observation, even though there are some opportunities for resolution maximizing ground-based observations of very limited regions of the sky, utilizing facilities such as the Atacama Large Milimeter Array. If the PAH abundance variations are not pronounced enough at the resolutions we use to trace the AME, then we will not see a correlation at all. Likewise, if there are enivonmental diffrences only visible at sub-degree angular scales, we will have a hard time controlling for the excitation factors of spinning dust emission. Without separating excitation factors from column density of the carriers
