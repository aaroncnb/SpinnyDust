\chapter{Conclusions}
  \label{ch:conclusions}

\section{Correlations }

  Using the presently available data, it is difficult to isolate the PAH-dust relationship, from the dust-AME relationship. We find throughout this work that a PAH-emission to AME correlation is readily demonstable, in intensity. Testing whether a second-order relationship exists, such that PAH emission is a better predictor of the AME than thermal dust emission, has proven inconclusive. One exception is that of $\lambda$~Orionis.

  \subsection{Variance between all-sky and $\lambda$~Orionis results}
      In $\lambda$~Orionis, our results suggest that PAH mass correlates better with AME than does the total dust mass. This point has not been demonstrated before with multi-wavelength dust SED calculations, taking into account calibration uncertainty, detector noise. Nevertheless we emphasize this point cautiously, because we were not able to generalize it to a larger scale (all-sky data with, effectively, a mask on the galactic plane and high latitude emission.)

      In Ch.~\ref{ch:allsky} we have demonstrated the presence of factors that, if large enough, could dilute such a correlation if it exists (under-subtraction of free-free emission in the AME map, Figs.~\ref{fig:AMEvartoDust_ffandSyncCountours} and~\ref{fig:AMEfixtoDust_ffandSyncCountours}). Thus we are simply not able to confirm or refute a ``spinning PAH'' hypothesis for the source of AME. However with at least the localized result within the $\lambda$~Orionis reigon, we suggest that there is ample reason to continue testing such a hypothesis, in localized studies.

      Results presented here do not conflict with results from Ch.~\ref{ch:lori}, except in that the findings from Fig.~\ref{fig:bootstrap_vs_AME}, wherein $r_{p}$ of A9 emission correlates better with AME than P545 emission, do not generalize to the results of this Ch.~\ref{ch:allsky} (either the masked or unmasked cases.) However, in the masked all-sky comparison as well as in $\lambda$~Orionis, we find a better corrleation between A9 and AME than between I12 and AME when we subject the correlations to bootstrap resampling.

      Interestingly, the ring-structure of $\lambda$~Orionis region analyzed in Ch.~\ref{ch:lori}, does not have any apparent conterpart in the $AME/R$ ratio maps in Figs.~\ref{fig:AMEfixtoDust_ffandSyncCountours}~and~\ref{fig:AMEvartoDust_ffandSyncCountours}. The difference results between Ch.~\ref{ch:lori} and Ch.~\ref{ch:lori} may be explained by variations in the component separation reliability. It is apparent that with the presently available data, there are a very limited number of regions on the sky which do not show strong free-free or synchrotron emissions, while also having enough S/N in the MIR to reliably probe a relationship between PAH abundance fluctuations and AME fluctuations. $\lambda$~Orionis may be one of the exceptions, where the S/N high enough, and we are in fact able to see an improved correlation between PAHs and AME, relative to the overall dust emission to AME connection.

      We also disucssed in Ch.~\ref{ch:allsky} the possibility that the PAH emission may not necessarily be optically thin. If this were true, it may explain discrepancies for our $\lambda$~Orionis result and the results in this chapter. $\lambda$~Orionis provides us a relatively clean line of sight compared to delocalized all-sky analysis.

      \subsection{Implications of an absent PAH-AME correlation}
      In the case that we are able to rule PAHs out as the AME carrier with confidence, this may imply some constraints on the PAH size distribution and dipole moment. As discussed in the theoretical work by \cite{draine98a, ali-haimoud10} and others, as well as the observational work by \cite{hensley16}, if PAHs exist in the ISM, they are very likely to be spinning rapidly. Thus if are able to confirm that they do not produce AME, this could tell us the morphology of ISM PAHs are not in an appropriate range to produce the observed AME. Spinning PAHs may be a reality even in the case that the AME comes from something else, if the PAHs are too large or lack appropriate electric dipoles.

    The fact that we do not see a clear preferential relationship between AME and PAHs in the all-sky analysis, does not rule out a contribution by spinning PAHs to the AME. Likewise the fact that we do see a better correlation in $\lambda$~Orionis does not rule out contributions from other proposals, such as that form non-PAH small grains, or nanosilicates.


\section{Future Works}
  New tests should: push limitations in the available data and, focus on regions well-documented to show a spinning-dust-like AME spectrum. While a simple excess of microwave emission may be able to be fit be a spinning dust model SED, the spinning dust explanation is not testable unless the observed SED shows evidence of a low-frequency downturn. This is suggested by our al-sky analysis, wherein many of the fitted peak frequencie of the AME are outside of observational microwave constraints.

  We note two major limitations in the current state of AME, and in general, all multi-wavelength analysis.
  The first is data limitations, the second is a lag in analysis innovations.

  In terms of data, there are two major boundaries that, based on results presented here, must be pushed. Spatial resolution and spectral coverage.
  Spatial resolution constraints are a critical issue in multiwavelength astornomy, especially in dust-related works. To explain why, simply consider the description of dust research from Ch.~\ref{ch:intro}, and Tab.~\ref{tab:data}. The spatial resolution of available data has a wide discrepancy from the MIR to microwave: ~10 arcseconds for AKARI 9 micron to at least a degree FWHM for the ``effective resolution'' of parameter maps. Likewise, if there are enivonmental diffrences only visible at sub-degree angular scales, we will have a hard time controlling for the excitation factors of spinning dust emission. Without separating excitation factors from column density of the carriers. Thus opportunities for resolution maximizing ground-based observations of very limited regions of the sky, utilizing facilities such as the Atacama Large Milimeter Array, and pairing these will a detailed assesment of tracers of small dust grains, will be helpful.

  Future breakthroughs in the AME may be seen if we are able to increase the number of regions with a reliable AME estimation. This can be obtained through improved synchrotron emission constraints, at higher resolutions. C-Band All Sky Survey (CBASS) at 5~GHz will be helpful in this regard \citep{irfan15}. C-BASS is expected to provide higher resolution low frequency constraints for the whole sky (48$''$ compared to 56$''$ for $H408$), with improved sensitivity (0.1$\mu$K). In terms of localized studies, it may be fruitful to consider the physical environment of each region (i.e. ionization state, gas temperature and density, other conditions indicated in \cite{draine98a, ali-haimoud10} to affect the SED shape) rather than relying of frequency-shifts to a template spectrum. More detailed exploration of PAH ionization, and other fluctuations between individual PAH features relative to variations in the AME spectral profile may also help us understand potential roles of PAHs in producing AME. However, even if we have excellent constraints on PAH emission features, synchrotron emission, free-free emission, and a powerful spinning dust SED model, the key data will be spectral coverage of the full AME profile. As noted in Ch.~\ref{ch:datasources}, the all-sky maps currently available, do not give constraints in the 10-20~GHz range. This means that lower frequency rspinning dust peaks and emissiviities are not well known. Improved understanding of AME and spinning dust therefore will require observations such as those being undertaken by the Q-U-I JOint Tenerife Experiment (QUIJOTE) project, which offers coverage between 10 and 20~GHz \citep{santos15}.
