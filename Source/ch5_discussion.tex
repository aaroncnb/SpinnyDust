\chapter{Discussion}
  \label{ch:discussion}

    We have compared AME with infrared dust emission from 2 approaches. The results and discussion contained here apply to an angular scale of approximately 1$^{\circ}$. In general, our results support an AME-from-dust hypothesis. We do not find evidence that the AME is exclusively carried by PAHs, but we do find evidence that the AME correlates with the AKARI 9 micron band emission better than other MIR bands- and better than all other IR bands sampled, when looking at the particularly prominent AME region, $\lambda$~Orionis. The results may be consistent with a scenario in which PAH mass, cold dust, and the AME are all tightly correlated.

\section{AME:$T$, $G_{0}$}

    According to spinning dust theory outlined in \cite{draine98a} and in subsequent works by \cite{ysard10a}, the AME profile and intensity will depend in part on the ISRF- but as is well-stated in \cite{hensley17a}, exactly how the ISRF will affect the AME SED is a more complicated question. Absorbed starlight photons may be able to rotationally excite the carriers, but if an enhanced ISRF leads to increased dust heating, then the increased IR emission can rotationally de-excite the carriers. However the ISRF affects not only the dust temperature but ionization of the carriers.


\section{Microwave foreground component separation}

    There are known degeneracies between the foreground parameters of the COMMANDER maps (spinning-dust, and free-free, synchrotron components as described in \cite{planck15X}.) This can be demonstrated by comparing the ratio map of the PCXV intensity to thermal dust intensity.
