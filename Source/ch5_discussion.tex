\chapter{Summary}
  \label{ch:Summary}

  In Ch.~\ref{ch:intro} we demonstrated the both the mysteries and potential for answers revealed by all-sky observation and analysis. A particular mystery, that of the Anomalous Microwave Emission, and the popular but yet unproven hypothesis that this \gls{ame} comes from rapidly spinning tiny dust grains, perhaps \gls{pah}s, was introduced. We explained the merit of testing a spinning \gls{pah} hypothesis, while noting that spinning nanosilicates, or even magnetic dipole emission from dust have also been put forth in the literature. We also introduced the fundamental statistical methods used in this work, building towards an explanation of and justification for more advanced techniques like bootstrap testing, and the key method used by this work: \gls{hb} analysis.

  In Ch.~\ref{ch:datasources}, we overviewed a collection of infrared all-sky surveys that probe the dust \gls{sed} from the \gls{pah} range to the \gls{fir}, and discussed how these could help explore the \gls{ame} question. We described also complimentary data and parameter maps from the Planck Collaboration that can be compared with the IR maps. Particular advantages of the \gls{pah} tracing bands, especially the A9 band, for covering not only neutral but potentially charged \gls{pah}s, were discussed in the context of \gls{ame} investigation. Limitations were given for all of these data sets--- most of these boiling-down to component separation either of the Zodical light from thermal dust emission, or to the latter from other microwave foregrounds and the CMB.

  In Ch.~\ref{ch:lori} we combined and processed the data presented in Ch~\ref{ch:datsources} to investigate $\lambda$~Orionis a sky with a resolved shape in all of the data, and demonstrating strong \gls{ame} as demonstrated by \cite{planck15}. We directly explore the relationship between \gls{pah} emission and the \gls{ame} in this region.  We evaluated the background, noise, and contamination from systematic errors and point sources in these data, and compared them on a common resolution and grid. All of the analyses from simple correlation plots, to bootstrap analysis, to dust \gls{sed} fitting suggest that A9 emisison correlates with \gls{ame} better than I12 or D12- supporting \gls{pah} emisison as the source of \gls{ame}, and suggesting the role of charged \gls{pah}s should be further explored. In addition we find that the bands near the thermal dust emisison peak show a weaker correlation, perhaps suggesting that harsh radiation fields may be destroying \gls{pah}s, leading to weaker \gls{ame}. We also cautioned that although \gls{pah} mass correlates better than dust mass, we still see a good correlation between FIR emisison and \gls{ame} (P545, P857). This raises the possibility that \gls{ame}, \gls{pah}s, and cold dust are all correlated. The central result of this chapter, and this thesis, is found when we compare the correlation strengths of PAH mass and total dust mass vs. the AME intensity. We show clearly via \gls{hb} that PAH mass exhibits a significantly stronger correlation with AME intensity than that of total dust mass vs. We demonstrated also that this result, with the present data and processing, could not have been revealed by \gls{lsm} analysis. \gls{hb} fitting such as that employed here, deveoped by \cite{galliano18}, should be considered as a standard part of the toolbox when tackling interstellar-dust related mysterious.

  In Ch.~\ref{ch:allsky} we attempted to find evidence that the result from Ch.~\ref{ch:datasources} applies even when looking at a less region-specific scale, using an all-sky analysis. However neither the full-sky unmasked case, or the case employing a mask of low S/N regions (mainly high galactic latitudes), point sources, and the galactic plane, showed a stronger correlation between \gls{pah} related emission and \gls{ame} than that seen between FIR emission and the \gls{ame}. We did however find that as in Ch.~\ref{ch:lori}, A9 tends to correlate with \gls{ame} better than I12 or D12. We discussed apparent systemtic issues with the \gls{ame} data, such as the possibility for under or over subtraction of free-free or synchrotron emission, and how these could serve to weaken evidence of any \gls{pah} or small grain correlation wiht the \gls{ame}. Scaling the infrared intensities by $U$ does not change this result. We reproduce the result from \cite{hensley16} that there is not an overall correlation between \gls{pah} fraction and \gls{ame}-per-dust-emisison, but expand this calculation showing that for a few limited regions on the sky there may be evidence of a correlation. We discuss how a potential variation on the optical thickness between UIR bands, FIR emisison, among other factors, could lead do a different result between $\lambda$~Orionis and the all-sky analysis. We also discuss potential observional advances that could help explore the issue further.

  Overall, using the presently available data, it is difficult to isolate the \gls{pah}-dust relationship, from the dust-AME relationship. We find throughout this work that a \gls{pah}-emission to \gls{ame} correlation is readily demonstable, in intensity. A test of whether a second-order relationship exists, such that \gls{pah} emission is a better predictor of the \gls{ame} than thermal dust emission, was shown to be inconclusive. One exception is that of $\lambda$~Orionis, where our results suggest that \gls{pah} mass correlates better with \gls{ame} than does the total dust mass. Nevertheless we emphasize this point cautiously, because we were not able to generalize it to a larger scale (all-sky data with, effectively, a mask on the galactic plane and high latitude emission.)  We are not able to confirm or reject a ``spinning \gls{pah}'' hypothesis for the source of \gls{ame}. However in $\lambda$~Orionis, and perhaps other regions yet to be examined in with dust \gls{sed} fitting, we suggest that there is still ample reason to continue testing this hypothesis.
