\chapter{Conclusions}
  \label{ch:conclusions}

\section{Correlations }

  Using the presently available data, it is difficult to isolate the PAH-dust relationship, from the dust-AME relationship. We find throughout this work that a PAH-emission to AME correlation is readily demonstable, in intensity. Testing whether a second-order relationship exists, such that PAH emission is a better predictor of the AME than thermal dust emission, has proven inconclusive. One exception is that of $\lambda$~Orionis--- PAH mass correlates better with AME than does the total dust mass. While this point has not been demonstrated before with multi-wavelength dust SED calculations, taking into account calibration uncertainty, detector noise. We emphasize this point cautiously however, because we were not able to generalize the result to a larger scale (all-sky data with, effectively, a mask on the galactic plane and high latitude emission.) While we have demonstrated the presence of factors that, if large enough, could dilute such a correlation if it exists (under-subtraction of free-free emission in the AME map), we are simply not able to confirm or refute a ``spinning PAH'' hypothesis for the source of AME. However with at least the localized result within the $\lambda$~Orionis reigon, we suggest that there is ample reason to continue testing such a hypothesis, in localized studies.

\section{Future Works}
  New tests should: push limitations in the available data and, focus on regions well-documented to show a spinning-dust-like AME spectrum. While a simple excess of microwave emission may be able to be fit be a spinning dust model SED, the spinning dust explanation is not testable unless the observed SED shows evidence of a low-frequency downturn. This is suggested by our al-sky analysis, wherein many of the fitted peak frequencie of the AME are outside of observational microwave constraints.

  We note two major limitations in the current state of AME, and in general, all multi-wavelength analysis.
  The first is data limitations, the second is a lag in analysis innovations.

  In terms of data, there are two major boundaries that, based on results presented here, must be pushed. Spatial resolution and spectral coverage.
  Spatial resolution constraints are a critical issue in multiwavelength astornomy, especially in dust-related works. To explain why, simply consider the description of dust research from Ch.~\ref{ch:intro}, and Tab.~\ref{tab:data}. The spatial resolution of available data has a wide discrepancy from the MIR to microwave: ~10 arcseconds for AKARI 9 micron to at least a degree FWHM for the ``effective resolution'' of parameter maps. Likewise, if there are enivonmental diffrences only visible at sub-degree angular scales, we will have a hard time controlling for the excitation factors of spinning dust emission. Without separating excitation factors from column density of the carriers. Thus opportunities for resolution maximizing ground-based observations of very limited regions of the sky, utilizing facilities such as the Atacama Large Milimeter Array, and pairing these will a detailed assesment of tracers of small dust grains, will be helpful.

  \subsection{Analysis constraints}
  If fluctuations in AME carrier relative abundance are not pronounced enough at the resolutions we use to trace the AME, we may miss a correlation with AME intensity-per-dust fluctuations. This applies to PAHs or any other carrier of the AME. Thus it will be useful to construct theoretical predicitons of the magnitude of such fluctuations, and convolve these with predicted effects of noise both in the measurements of dust emission and in the AME.

  Another limitation is 
