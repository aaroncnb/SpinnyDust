\section{Discussion}
  \label{ch:discussion}

  We have compared AME with infrared dust emission from 2 approaches. The results and discussion contained here apply to an angular scale of approximately 1$^{\circ}$. In general, our results support an AME-from-dust hypothesis. At 1$^{\circ}$ angular resolution, we do not find evidence that the AME is exclusively carried by PAHs. The correlation of far IR dust emission with AME appears to be the best predictor, with MIR/PAH emission being marginally more weakly correlated with AME. As far as the MIR bands are concerned, at least in terms of intensity, $I_{9\mu{}m}$ shows the strongest correlation. This result is mirrored in LOri, however in that case $I_{9\mu{}m}$ and $I_{545 \mu{}m}$ show the strongest correlation. In the case of $\lambda$~Orionis, AME vs. PAH emission is at least as strong as AME vs. FIR emission.

      \subsection{AME:Dust}

        As noted in Ch. \hyperref[ch:intro]{\ref*{ch:intro}}, previous studies found that the AME generally correlates at dust-related IR wavelengths \citep{ysard10b,planckXV, hensley16}. We see the same overall pattern in the present study, for both the all-sky pixel-based analysis, and the inspection of the $\lambda$~Orionis region.

         In our all-sky comparison, we find a first-order correlation between IR intensity and AME intensity, for each of the 12 wavelengths sampled. This is again consistent with the previous investigations of the AME cited above, in that the FIR emission shows the tightest correlation with the AME intensity.

         In testing for a second-order correlation, we divided the IR intensities and AME intensity by the dust radiance, and again performed the band-by-band all-sky comparison. There is evidence of a residual correlation between $I_{MIR}$ and $I_{AME}/R$. Unsurprisingly, the strong correlation between $I_{FIR}$ and $I_{AME}$ disappears when scaling by $R$, as the the FIR bands are dominated by thermal dust emission. In this case, we again find no evidence of an improved correlation for the PAH-dominated bands.

        %In the case of the AME candidate regions, do we find evidence of a statistically stronger relationship between AME and total dust mass than AME vs. PAH mass, for the full set of 98 regions. We were able to confirm that the correlation coefficients are statistically distinct, via boostrap re-sampling (with replacement.) However the difference is marginal. Moreover the full set of 98 regions includes many regions that are either near the galactic plane, or do not have a strong S/N for the AME component separation. A similar result is found when we compare AME to the total dust luminosity and PAH luminosity.

        The closeness of the correlation coefficients found here is consistent with the results of the IRAS vs. AME correlation test result from \cite{planckXV}. They found that the correlation coefficient among the 4 IRAS bands (12, 25, 60, and 100~$\mu$m) differ from one another only by about 5\%, across the whole set of 98 regions. The trend of AKARI MIR and FIR data vs. the AME does not disagree with their IRAS comparison. This work adds that bands longer than IRAS 100~$\mu$m also correlate strongly with AME, especially the two Planck/HFI bands used.

      \subsection{AME:PAH}

        In the case of $\lambda$~Orionis we found that accross the whole region, AKARI 9~$\mu$m emission and Planck 545~GHz emission were the most strongly correlated with AME, having Spearman coefficients of 0.81 and 0.80. The results may be consistent with a scenario in which PAH mass, cold dust, and the AME are all tightly correlated. Weaker correlation from 25 to 70~$\mu$m may indicate that AME is weaker in regions of warmer dust and stronger radiation fields. This would be consistent with PCXV wherein highly significant AME regions tended to have a lower dust temperature than other regions. The fact that the correlation strengths of PAH-tracing mission the FIR emission are similar is in-line with what we have seen in \cite{ysard10b} and \cite{hensley16}. In those works, the two relationships (MIR vs. AME and FIR vs. AME) are very close, although these two papers are odds as to which relationship is stronger, and thus in their final interpretation. However in the investigation of the Perseus molecular cloud complex by \cite{tibbs11}, PAH emission (as well as VSG emission and hydrogen column density, $N_{H}$) does not show a strong correlation with AME compared to environmental paramters, such as dust temperature.

        On an all-sky basis, each of the bands sampled show correlation with the AME, however the FIR bands always show the strongest correlation. In fact, the correlation pattern of AME vs. each of the IR bands, very strongly resembles the correlation results of the Planck HFI bands vs. all of the other bands (Fig. \ref{fig:AME_IR_crosscorr_allbandsg}.) This is readily apparent from the pixel-density plots in Fig. \ref{fig:AMEvsDust_allsky_allbands}, wherein the FIR bands pixels show a very similar density profile vs. the AME. In attempting to factor out this first-order correlation, dividing the AME and IR intensities by the dust radiance for each pixel, we find the there is still a residual correlation between the MIR bands and the AME. The FIR bands scaled by the dust radiance, as expected, lack correlation with $I_{AME}/R$.


      \subsection{AME:$T$, $G_{0}$}

        According to spinning dust theory outlined in \cite{draine98a} and in subsequent works by \cite{ysard10a}, the AME profile and intensity will depend in part on the ISRF- but as is well-stated in \cite{hensley17a}, exactly how the ISRF will affect the AME SED is a more complicated question. Absorbed starlight photons may be able to rotationally excite the carriers, but if an enhanced ISRF leads to increased dust heating, then the increased IR emission can rotationally de-excite the carriers. However the ISRF affects not only the dust temperature but ionization of the carriers.

        \cite{hensley16} looked at the $AME$/$R$ ratio vs. $T$ and found only a slight anti-correlation of $P = -0.06$.

      \subsection{Microwave foreground component separation}

        There are known degeneracies between the foreground parameters of the COMMANDER maps (spinning-dust, and free-free, synchrotron components as described in \cite{planck15X}.) This can be demonstrated by comparing the ratio map of the PCXV intensity to thermal dust intensity.
