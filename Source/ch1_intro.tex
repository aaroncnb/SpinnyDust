
\chapter{Introduction}
  \begin{quotation}
    \small
    \textit{``It is now plain that about 75\% of the data we would like to have can be obtained from good ground-based sites''} \\
    -H. Johnson, 1966
  \end{quotation}

\section{All-sky Astronomy}

  \subsection{A brief history of sky maps}

    All-sky astronomy is not new. Indeed, the notion of capturing a particular ``object'' or ``source'' with a camera and saving it for later investigation would be completely alien to the first astronomers and astronavigators. Absence of telescopes forced us to describe the sky in terms of its larger patterns, brightest characters. What is new however is the notion of preparing an archive of the sky itself for not only the research whims of a single investigator, team, institute, or even a single nation- rather, all-sky surveys tend to be international endeavors in their production, and even more so in their utilization.

    \begin{table}
      \caption{A timeline of all-sky surveys}
      \label{tab:allsky-maps}
    \end{table}

  \subsection{Infrared astronomy}

  Infrared astronomy was essentially non-existant as recently as the 1950s \citep{johnson66} \footnote{or 1920s, if we judge by the first IR observations}. Compare this to visible wavelength- a field so old we name it after the bio-evolutionary advent of sight, itself. Even radio astronomy with its own logistical and technological challenges, has been around since at least 1932.


Despite the above claim from \citep{johnson66}, astronomers were apprently not content to be constrained by atmospheric IR windows, even from the best of ground-based sites. Or perhaps interests have shifted so dramtically since 1966, that all of the investigations enabled by rocket-based, space-based, even Boeing 747-based IR astronomy would have bored 75\% of astronomers in the '60s. The meaning of ``far infrared'' has even redshifted, so to speak, from the \cite{johnson66} definition of ``4 to 22 um'' \footnote{\footnotesize For our purposes, we consider the FIR to cover 60 to 550 microns, partially out of conveneince- FIR bands, in this paper, means the IRAS 60 and 100 micron, all four FIS bands, and the HFI 857 GHz and 545 GHz bands. The two IRC bands and the IRAS 12 and 25 micron bands we will refer to collectively as the MIR bands.}.

The ability to map and archive the sky with satellites - not only in the optical and infrared, but well into the microwave regime \footnote{\cite{johnson66} did not offer a definition of ``microwave", though \cite{penzias65} had stumbled intro microwave astronomy a year before. Perhaps space-based microwave astronomy was not in the percentage of data that 1960s astronomers would have liked. } -

\section{Scope of this Dissertation}

The main goal of this work is to highlight a particular application of multiwavelength (mid-IR to radio) all-sky data analysis. We describe the interrelatedness between mid to far IR dust emission and possible microwave emission from dust. This is accomplished through an investigation of phometric all sky maps mainly from AKARI, IRAS, and Planck. We do not explore the modeling of microwave dust emission itself, rather the comparison of archival data and parameter maps. Modeling of the exact physical mechanism of the anomalous component of galactic microwave foreground emission from first principles is well beyond the scope of this work. We consider this problem first on an all-sky basis, not focusing on any pre-selected object of the sky - however we do use various systematic masks, i.e. of the ecliptic plane to avoid zodiacal light contamination. We then focus on a region highlighted by the Planck Collaboration as being especially worthy of further investigation, and has a resolvable topology even at 1-degree resolution. Essentially all of the analyses and conclusions presented in this work apply only to an angular scale of approximately 1-degree. We make no claims from the outset that the data sources provide a definitive measurement of the abundance of PAHs or of the carrier(s) of AME.

This dissertation is accompanied by a github repository stored at: https://github.com/aaroncnb/CosmicDust. Virtually all of the analyses code are available in that repository, in the form of Jupyter notebooks (along with the figures and the code used to generate them.) The dust SED fitting code is not part of that reposittory, but is described in Galliano et al. (in prep.)

\section{Anomalous Microwave Foreground}

     Since its first detection in early microwave observations, the anomalous microwave emission/foreground (AME) has been found to be a widespread feature of the microwave Milky Way \citep{dickinson13r}. \cite{kogut96,deoliveiracosta97,leitch98} showed that the AME correlates very well with infrared emission from dust, via COBE/DIRBE and IRAS far-IR maps. However there remains much mystery, except that the most likely source of the AME is interstellar dust \citep{ysard10a,tibbs11,hensley16}. Let alone uncertainty in the physical mechanism of the AME, and even assuming a dusty origin- we are still puzzled as to the chemical composition and morphology of the carrier(s).

     From the observed spatial correlation between AME and dust emerged two prevailing hypotheses:
    1) Electric dipole emission by spinning small dust grains, a mechanism proposed in \cite{erickson57} and \cite{hoyle70}, with further discussion in \cite{ferrara94}. \cite{draine98b} give the earliest thorough description, with substantial updates contributed more recently by \cite{ysard10a}, \cite{ali-haimoud09}, \cite{hoang10} and several others. \cite{hensley17a} propose that such small spinning grains may consist primarily of silicates, and that this is allowed by observational upper-bounds of nanosilicate abudance, although nanosilicates have not yet been detected in the ISM. \cite{dickinson13r} provide a detailed overview of AME and spinning dust literature. \\
    2) Magnetic dipole emission, caused by thermal fluctuations in grains with magnetic inclusions, proposed by \cite{draine99}.
     More recently, modeled spectra for potential candidate carriers have appeared in the literature: PAHs, grains with magnetic inclusions \citep{draine13, ali-haimoud14, hoang16a}.\\

    A third, but not widely accepted, possible explanation for AME is discussed in \cite{jones09}. They have suggested that the emissivity of dust, in the spectral range related to AME, could contain features caused by low temperature solid-state structural transitions.

     Although spinning dust need not be the only emission mechanism, the photometric signature of the AME, has so far been commonly explained via spinning dust parameters \citep{ysard11,ali-haimoud10}. We explore the case that the AME signature arises from spinning dust emission. If the AME is carried by spinning dust, the carrier should be small enough that it can be rotationally excited to frequencies in the range of 10-40~GHz, and must have a permanent electric dipole. Within contemporary dust SED models, only the polycyclic aromatic hydrocarbon family of molecules (PAHs), or nanoscale amorphous carbon dust fit these criteria. Those PAHs which have a permanent electric dipole (i.e. coranulene, but not symmetric molecules like coronene), can emit rotationally. However the carrier need not be carbon-based. Indeed, \cite{hensley17a} claim that AME can be explained without carbonaceous carriers, using only spinning nanosilicates.

     While neither nanosilicates nor any particular species of PAHs have been conclusively identified in the ISM, mid-infrared features associated with PAH-like aromatic materials have been observed. In fact, ``the PAH features'' are ubiquitous in the ISM \citep{giard94,onaka96,onaka00}, such that the carriers must be abundant. \cite{andrews15} strongly argue for the  existence of a dominant ``grandPAH'' class, containing 20 to 30 PAH species. There has yet to be any detection of features related to nanosilicates. There is only an upper-bound from IRTS observations of \cite{onaka96} and calculations by \cite{li01}.

     Assuming a rotational emission model, the AME signature (consistent with continuum emission having a peak between 15 and 50~GHz ) implies very small oscillators (\textasciitilde{}10~nm). In any case, the PAH class of molecules are the only spinning dust candidate so far which show both: \\
     1) Evidence of abundance in the ISM at IR wavelengths, and \\
     2) A predicted range of dipole moments (on order of 1~debye), to produce the observed AME signature \citep{draine98b, lovas05, thorwirth07}. However, it should be noted that the current upper-bound on the abundance of nanosilicates, allows for a "spinning nanosilicate" explanation for the AME, as shown by \cite{hensley17a}. Due to the (apparently) continuous shape of the AME SED, a spinning dust explanation requires a distribution of dipole moments and/or rotational velocities of the carriers. Of course, the AME cannot simply be modeled by a distribution of carriers. Environmental factors affecting the rotational excitation of the carriers must be considered.

     In the spinning dust model, there are several possible excitation factors for spinning dust. While the question of whether or not such small grains would be spinning at all is trivial- a case in which grains had no angular momentum would be bizarre- for the grains to have rotational velocities high enough to create the observed AME, they must be subject to strong excitation mechansisms. The dominant factors that would be giving grains their spin, are broken down by \cite{draine11} into basically two categories: 1) Collisional excitation. 2) Radiative excitation, the sum of which could lead to sufficient rotational velocities for sufficiently small grains. However the extent of excitation will depend on environmental conditions, i.e. there will be more frequent encounters with ions and atoms in denser regions (so long as the density is not high enough to coagulate the small grains), and more excitation due to photon emission with increasing ISRF strength \citep{ali-haimoud09, ali-haimoud13}. One of the strongest potential excitation mechansims listed in \cite{draine11} is that of negatively charged grains interacting with ions. Thus not only must we consider environmental factors, grain composition and size, but also the ionization state of the carriers. (For example, ionizaed vs. neutral PAHs.) The dependence of the observed AME on ISM density is modeled by \cite{ali-haimoud10}, demonstrating that denser regions may have a stronger AME component (although it can be observationally challenging to resolve dense vs. diffuse AME producing regions.)

    The overall pattern among large-scale studies seems to show that all of the dust-tracing photometric bands correlate with the AME (and each other) to first-order.  On an all-sky, pixel-by-pixel basis, at 1$^{\circ}$ angular resolution, \cite{ysard10b} find that 12~$\mu$m emission, via IRAS, correlates slightly more strongly with AME (via WMAP) than with 100~$\mu$m emission.  They also find that scaling the IR intensity by the interstellar radiation field strength (given as $G_0$, a measure of ISRF relative to that of the solar neighborhood) improves both correlations. THey interpret this finding as evidence that AME is related to dust, and more closely related to the small stochastically emitting dust that is traced by 12~$\mu$m emission.

    However in a similar work, \cite{hensley16} report that the 12~$\mu$m emission (via WISE) correlates less tightly with AME than with thermal dust radiance, using the Planck Collaboration dust and AME component-separation maps \citep{planck15X}. Also at odds with \cite{ysard10b}, they report that AME correlates more strongly with 12~$\mu$m intensity than with the intensity scaled by the interstellar radiation field. They interpret this as AME and PAH emission both being correlated with the total dust radiance, but that there is no preferential relationship between PAHs and the AME.

     The story is no more clear when looking at the average properties of individual regions. \cite{planckXV} find that among 22 high-confidence ''AME regions" (galactic clouds such as the $\rho$~Ophiuchus cloud and the Perseus molecular cloud complex) AME vs. 12~$\mu$m  shows a marginally weaker correlation than AME vs. 100~$\mu$m (via IRAS). \cite{tibbs11} examined the AME-prominent Perseus Molecular Cloud complex, finding that while there is no clear evidence of a PAH-AME correlation, they do find a slight correlation between AME and  $G_0$.

     In this work, we attempt to reach some stronger consensus on the large-scale AME vs. IR-dust story, keeping in mind that resolution limitations are a could dilute more subtle connections between potential connections between dust IR emission and the anomalous foreground. In Section \ref{sec:data}, we describe the all-sky surveys and component maps used in this paper. The sources range from PAH-domianted MIR bands from AKARI and IRAS to FIR and microwave-derived maps from Planck Observatory. These bands are listed in table \ref{sec:data}, and are outlined in the following Data section. Section  \ref{sec:analysis} describes our investigation approach: looking at all-sky trends, a circular aperture photometry on a list of regions of interest, and a localized study of the AME region of largest angular size, lambda Orionis. Section \ref{sec:discussion} describes the conclusions of these 3 approaches, and compares them to previous AME vs. dust emission studies.

    We utilize the newly created AKARI/IRC 9~$\mu$m all-sky map, which more completely covers PAH features than previous all-sky surveys, and has the highest fractional contribution from PAH emission, as can be seen in Fig. \ref{fig:inband_ionfrac_bar}, assuming a PAH and grain SED model such as \cite{draine01}. This is especially true as the ISRF intensifies, and the IRAS or WISE 12~$\mu$m bands include more continuum emission.
