All-sky astronomy is not new. Indeed, the notion of capturing a particular "object" or "source" with a camera and saving it for later investigation would be completely alien to the first astronomers and astronavigators. Absence of telescopes forced us to describe the sky in terms of its larger patterns, brightest characters. What is new however, is the notion of preparing an archive of the sky itself for not only the research whims of a single investigator, team, institute, or even a single nation- rather, all-sky surveys tend to be international endeavors in their production, and even more so in their utilization.

This is especially true in the context of infrared astronomy, a field which was essentially non-existant as recently as the 1950s \citep{johnson66} \footnote{1920s, if we judge by the first IR observations}. Compare this to visible wavelength- a field so old we name it after the bio-evolutionary advent of sight, itself. Even radio astronomy with its own logistical and technological challenges, has been around since at least 1932.

`` Johnson+ 1966 p.194: "It is now plain that about 75\% of the data we would like to have can be obtained from good ground-based sites"
% ...p.1 "The Far Infrared, from 4 to 22 um''

Despite the above claim from \citep{johnson66}, astronomers were apprently not content to be constrained by atmospheric IR windows, even from the best of ground-based sites. Or perhaps interests have shifted so dramtically since 1966, that all of the investigations enabled by rocket-based, space-based, even Boeing 747-based IR astronomy would have bored 75\% of astronomers in the '60s.
