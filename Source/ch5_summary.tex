\chapter{Summary}
  \label{ch:Summary}

  In Ch.~\ref{ch:intro} we demonstrated both the mysteries and potential for answers revealed by all-sky observation and analysis. A paticular mystery, that of the Anomalous Microwave Emission (AME), and the popular but yet unproven hypothesis that this AME comes from rapidly spinning tiny dust grains, perhaps PAHs, was introduced. We explained the merit of testing a spinning PAH hypothesis, while noting that spinning nanosilicates, or even magnetic dipole emission from dust have also been put forth in the literature.

  In Ch.~\ref{ch:datasources}, we overviewed a collection of infrared all-sky surveys that help us prove the dust SED from the UIR band range to the FIR, and discussed how these could help explore the AME question. We described also complimentary data and parameter maps from the Planck Collaboration that can be compared with the IR maps. Particular advantages of the PAH tracing bands, especially the A9 band, for covering not only neutral but potentially charged PAHs, were discussed in the context of AME investigation. Limitations were given for all of these data sets--- most of these boiling-down to component separation either of the Zodical light from thermal dust emission, or to the latter from other microwave foregrounds and the CMB.

  In Ch.~\ref{ch:lori} we combined and processed the data presented in Ch~\ref{ch:datasources} to investigate a particular region of the sky with relatively high S/N in all of the data, and demonstrating strong AME. We evaluated the background, noise, and contamination from systematic errors and point sources in these data, and compared them on a common resolution and grid. All of the analyses from simple correlation plots, to bootstrap analysis, to dust SED fitting suggest that A9 emisison correlates with AME better than I12 or D12 --- supporting PAH emisison as the source of AME, and suggesting the role of charged PAHs should be further explored. In addition we find that the bands near the thermal dust emisison peak show a weaker correlation, perhaps suggesting that harsh radiation fields may be destroying PAHs, leading to weaker AME. We also cautioned that although PAH mass correlates better than dust mass, we still see a good correlation between FIR emisison and AME (P545, P857). This highlights the fact that AME, PAHs, and cold dust are all correlated.

  In Ch.~\ref{ch:allsky} we attempted to find evidence that the result from Ch.~\ref{ch:datasources} applies even when looking at a less region-specific scale, using an all-sky analysis. However neither the full-sky unmasked case, or the case employing a mask of low S/N regions (mainly high galactic latitudes), point sources, and the galactic plane, showed a stronger correlation between PAH related emission and AME than that seen between FIR emission and the AME. We did however find that as in Ch.~\ref{ch:lori}, A9 tends to correlate with AME better than I12 or D12. We discussed apparent systemtic issues with the AME data, such as the possibility for under or over subtraction of free-free or synchrotron emission, and how these could serve to weaken evidence of any PAH or small grain correlation wiht the AME. Scaling the infrared intensities by $U$ does not change this result. We reproduce the result from \cite{hensley16} that there is not an overall correlation between PAH fraction and AME-per-dust-emisison, but expand this calculation showing that for a few limited regions on the sky there may be evidence of a correlation. We discuss how a potential variation on the optical thickness between UIR bands, FIR emisison, among other factors, could lead to a different result between $\lambda$~Orionis and the all-sky analysis. We also discuss potential observional advances that could help explore the issue further, such as improved low frequency constraints,or detailed analysis of the spatial variation of PAH features paired with high resolution AME observation.

  Overall, using the presently available data, it is difficult to isolate the PAH-dust relationship, from the dust-AME relationship. We find throughout this work that a PAH-emission to AME correlation is readily demonstable, in intensity. A test of whether a second-order relationship exists, such that PAH emission is a better predictor of the AME than thermal dust emission, was shown to be inconclusive. One exception is that of $\lambda$~Orionis, where our results suggest that PAH mass correlates better with AME than does the total dust mass. Nevertheless we emphasize this point cautiously. We were not able to generalize it to a larger scale (all-sky data with, effectively, a mask on the galactic plane and high latitude emission.)  We are not able to confirm or reject a ``spinning PAH'' hypothesis for the source of AME. However in $\lambda$~Orionis, and perhaps other regions yet to be examined in with dust SED fitting, we suggest that there is still ample reason to continue testing this hypothesis.
